For the studies presented, the formalism of the scattering amplitude is used
to describe the interactions of a boson $\PH$ with a pair of vector bosons $\V_1$ and $\V_2$.

\subsection{Spin-zero resonance}
For a spin-zero boson $\PH$ and two spin-one gauge bosons $\V\V$, such
as $\PZ\PZ, \PZ \Pgg, \Pgg\Pgg, \PW\PW$, or $\Pg\Pg$, the scattering amplitude
presents three invariant tensor terms with coupling complex constants $a_{i}^{\V\V}$
which in general can depend on the Lorentz invariant four-momenta
of $\V_1$ and $\V_2$ squared, $q_{\sss\V1}^2$ and $q_{\sss\V2}^2$. 
In the following, the terms up to $q_{\sss\V}^2$  are kept in the expansion under the assumption 
of small contributions from anomalous couplings
%
\begin{eqnarray}
A(\PH\V\V) \sim 
\left[ a_{1}^{\V\V} 
+ \frac{\kappa_1^{\V\V}q_{\sss\V1}^2 + \kappa_2^{\V\V} q_{\sss\V2}^{2}}{\left(\Lambda_{1}^{\V\V} \right)^{2}} \right] 
m_{\sss\V1}^2 \epsilon_{\sss\V1}^* \epsilon_{\sss\V2}^* \\ \nonumber
+ a_{2}^{\V\V}  f_{\mu \nu}^{*(1)}f^{*(2),\mu\nu} 
+ a_{3}^{\V\V}   f^{*(1)}_{\mu \nu} {\tilde f}^{*(2),\mu\nu}\,,
\label{eq:formfact-fullampl-spin0} 
\end{eqnarray}
%
where $f^{(i){\mu \nu}} =
\epsilon_{{\sss\V}i}^{\mu}q_{{\sss\V}i}^{\nu} -
\epsilon_{{\sss\V}i}^\nu q_{{\sss\V}i}^{\mu} $ is the field strength
tensor of a gauge boson with momentum $q_{{\sss\V}i}$ and polarization
vector $\epsilon_{{\sss\V}i}$, ${\tilde f}^{(i)}_{\mu \nu} =
\frac{1}{2} \epsilon_{\mu\nu\rho\sigma} f^{(i),\rho\sigma}$ is the
dual field strength tensor, the superscript~$^*$ designates a complex
conjugate, $m_{\sss\V1}$ is the pole mass of the vector boson $\PZ$ or
$\PW$, and $\Lambda_{1}$ is the scale of BSM physics and is a free
parameter of the model~\cite{Anderson:2013afp}. The tree-level SM-like
contribution corresponds to $a_{1}^{\PZ\PZ}\ne 0$ and $a_{1}^{\PW\PW}
\ne 0$, while there is no tree-level coupling to massless gauge
bosons, that is $a_{1}^{\V\V}= 0$ for $\PZ \gamma, \gamma\gamma$, and
$\Pg\Pg$. The other terms in the SM can be generated through loop
effects, and are expected to be small enough to be observed with the
current LHC dataset, thus they are considered as anomalous couplings.

The parity-conserving interaction of a pseudoscalar ($CP$-odd state)
corresponds to the $a_{3}^{\V\V}$ terms, while the other terms
describe the parity-conserving interaction of a scalar ($CP$-even
state).  The $a_{3}^{\V\V}$ terms appear in the SM only at a
three-loop level and receive a small contribution.  The $a_{2}^{\V\V}$
and $\Lambda_{1}^{\V\V}$ terms appear in loop-induced processes and
also give small contributions $O(10^{-3} - 10^{-2})$.

Contributions from BSM particles can change both the magnitude and the
phases of these couplings, given their non-trivial dependence on the
Lorentz invariant quantities. When the particles in the loops
responsible for these couplings are heavy in comparison to the Higgs
boson mass, the couplings are real.  The scenarios are parameterized
in terms of the effective fractional cross sections and their phases
with respect to the two dominant tree-level couplings $a_1$ and
$a_1^{\PW\PW}$ in the $\PH\to \V\V\to 4\ell$ and $\PH\to \PW\PW\to
\ell\nu\ell\nu$ processes, respectively:
$(f_{\Lambda1},\phi_{\Lambda1})$, $(f_{a2}, \phi_{a2} =
\mathrm{arg}\left(\frac{a_{2}}{a_{1}}\right))$, $(f_{a3},\phi_{a3} =
\mathrm{arg}\left(\frac{a_{3}}{a_{1}}\right))$.  The couplings of the
Higgs boson to $\PZ\Pgg$ and $\Pgg\Pgg$ are also accessible in these
decays and can be measured with the same techniques in the
$\PH\to4\ell$ decays, but with the current LHC dataset they are much
better constrained via the decays on the on-shell gauge bosons.

The couplings in the $\chanHZZ$ and $\chanHWW$ decay channels can be
related with functions of two free parameters. If $f_{ai}$ is the
fraction in the $\PH\PZ\PZ$ coupling, then the other parameter can be
expressed in terms of the ratio of the anomalous couplings of the two
channels:
%
\begin{eqnarray}
r_{ai} = \frac{a_i^{\PW\PW} / a_1^{\PW\PW}  }{  a_i / a_1}\,, ~~{\rm or}~~~ R_{ai} = \frac{ r_{ai} |r_{ai}| }{  1 + r_{ai}^2 }  \, .
\label{eq:ratio_ww_zz}
\end{eqnarray} 
%
%% and the combination of $f_{ai}$ in the two channels is viable through the relationship:
%% %
%% \begin{eqnarray}
%% f_{ai} = \left[ 1+r_{ai}^2(1/f_{ai}^{\PW\PW}-1)\sigma_{i}^{\PW\PW}\sigma_{1}/(\sigma_{1}^{\PW\PW}\sigma_{i}) \right]^{-1}
%% \,.
%% \label{eq:a2_conversion}
%% \end{eqnarray}
%% %
%
A more complete description of the phenomenology of $\PH\V\V$
anomalous interactions can be found in Ref.\cite{CMS:2014gga}.


\subsection{Exotic spin-one and spin-two resonance}
A spin-one resonance cannot decay into $\Pgg\Pgg$ final state because
of the Landau-Yang theorem. We anyway tested this hypothesis for the
$\PZ\PZ$ and $\PW\PW$ channels, assuming the existence of two states
that decay in different channels. We test the spin-two hypothesis for
all the three channels.  The scattering amplitude of the exotic boson
with spin one ($X_{J=1}$) consists of two independent terms, which can
be written as
%
\begin{eqnarray}
A(\X_{J=1} \V\V) \sim b_{1}^{\V\V}  \left[ \left(\epsilon_{\sss\V1}^{*}q\right)\left(\epsilon_{\sss\V2}^{*}\epsilon_{\sss\X}\right) +
\left(\epsilon_{\sss\V2}^{*}q\right)\left(\epsilon_{\sss\V1}^{*}\epsilon_{\sss\X}\right) \right] + \\ \nonumber 
b_{2}^{\V\V}  \epsilon_{\alpha\mu\nu\beta}\epsilon_{\sss\X}^{\alpha}\epsilon_{\sss\V1}^{*\mu}\epsilon_{\sss\V2}^{*\nu}{\tilde q}^{\beta} \,,
\label{eq:ampl-spin1} 
\end{eqnarray}
%
where $\epsilon_{\sss\X}$ is the polarization vector of the boson $\X$
with spin one, ${q}=q_{\sss\V1}+q_{\sss\V2}$ and ${\tilde
  q}=q_{\sss\V1}-q_{\sss\V2}$~\cite{Gao:2010qx, Bolognesi:2012mm}.
Here the $b_{1}^{\V\V} \neq 0$ coupling corresponds to a vector
particle, while the $b_{2}^{\V\V}\neq 0$ coupling corresponds to a
pseudovector particle.  As in the case of spin-zero resonance, we
define a continuous parameter that describes the presence of the
corresponding terms $b_{1}^{\V\V} $ and $b_{2}^{\V\V}$ as an effective
fractional cross section $f_{b2}^{\V\V}$.  The $f_{b2}^{\V\V}$
parameter is used to test if the data favors the SM Higgs boson scalar
hypothesis or some particular mixture of the vector and pseudovector
states.

The scattering amplitude for a spin-two boson is more complex and its
expression can be found in \cite{CMS:2014gga}.  It contains ten
complex terms, and they are fully tested in this study.  In this case
we consider both the decays into massive gauge bosons, $\PZ\PZ$ or
$\PW\PW$, and to two on-shell photons, $\X\to\Pgg\Pgg$. Both $\qqbar$
production and gluon fusion, spin-two state are considered for the
$\PH\to4\ell$ final states. The set of models considered are: $2_m^+$,
$2_{h2}^+$, $2_{h3}^+$, $2_h^+$, $2_b^+$, $2_{h6}^+$, $2_{h7}^+$,
$2_h^-$, $2_{h9}^-$, $2_{h10}^-$.  The subscripts $m$ (minimal
couplings), $h$ (couplings with higher-dimension operators), and $b$
(bulk) distinguish different scenarios.  In the case of the $\Pgg\Pgg$
decay only the results for a massive graviton-like boson, $2_m^+$ are
considered.
