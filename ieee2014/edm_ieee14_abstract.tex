\documentclass[11pt]{article}
\usepackage{multicol}

\usepackage{abstract} % Allows abstract customization
\renewcommand{\abstractnamefont}{\normalfont\bfseries} % Set the "Abstract" text to bold
\renewcommand{\abstracttextfont}{\normalfont\small\itshape} % Set the abstract itself to small italic text

\usepackage{titlesec} % Allows customization of titles
\renewcommand\thesection{\Roman{section}} % Roman numerals for the sections
\renewcommand\thesubsection{\Roman{subsection}} % Roman numerals for subsections
\titleformat{\section}[block]{\large\scshape\centering}{\thesection.}{1em}{} % Change the look of the section titles
\titleformat{\subsection}[block]{\large}{\thesubsection.}{1em}{} % Change the look of the section titles

%%%%%%%%%%%%%%%%%%%%%%%%%%%
\hoffset = -2.0 cm
\voffset = -3.0 cm
\textheight = 25.0 cm
\marginparwidth = 1.0 cm
\evensidemargin = 1.0 cm
\textwidth =16.1 cm
%%%%%%%%%%%%%%%%%%%%%%%%%%%

%\def\bibname{ } 
%\def\nextref{\global\advance\refno by 1 \number\refno\relax}
%

%----------------------------------------------------------------------------------------
%	TITLE SECTION
%----------------------------------------------------------------------------------------

\title{\vspace{-15mm}\fontsize{14pt}{8pt}\selectfont\textbf{The CMS electromagnetic calorimeter calibration and  performance during LHC Run I and expectations for Run II}} % Article title

\author{
\large
\textsc{Emanuele Di Marco}\thanks{ on behalf of CMS Collaboration}\\[2mm] % Your name
European Organization for Nuclear Research (CERN) \\ % Your institution
emanuele.di.marco@cern.ch % Your email address
\vspace{-5mm}
}
\date{}

\begin{document}

\maketitle % Insert title
%\thispagestyle{fancy} % All pages have headers and footers

%----------------------------------------------------------------------------------------
%	ABSTRACT
%----------------------------------------------------------------------------------------

\begin{abstract}
\noindent The CMS ECAL is a high-resolution, hermetic, and homogeneous
electromagnetic calorimeter made of 75,848 scintillating lead
tungstate (PbWO$_4$) crystals. It relies on precision calibration in
order to achieve and maintain its design performance. A set of
inter-calibration procedures is carried out to normalize the
differences in crystal light yield and photodetector response between
channels. Different physics channels such as low mass di-photon
resonances, electrons from W and Z decays and the azimuthal symmetry
of low energy deposits from minimum bias events are used. A laser
monitoring system is used to measure and correct for response changes,
which arise mainly from the harsh radiation environment at the
LHC. The challenges of the different calibration techniques are
discussed along with the performance evolution during Run I. The
impact on physics is illustrated through the successful quest for the
Higgs boson via its electromagnetic decays, and the subsequent mass
measurement of the newly discovered particle. Conclusions are drawn
for the performance to be expected from 2015 onwards, following the
start of the LHC Run II.
\end{abstract}

\begin{multicols}{2}

\section{Introduction}

The electromagnetic calorimeter (ECAL) of the Compact Muon Solenoid
(CMS) is a hermetic, fine grained homogeneous crystal calorimeter
designed to achieve excellent energy resolution in a wide range of
photon and electron momenta. One of the primary goals of CMS is the
discovery of the Standard Model (SM) Higgs boson via its decays to
di-bosons.  In particular the sensitivity of the decay to two photons
is driven by the invariant mass resolution, and the sensitivity in the
four lepton by the capability to reconstruct and identify electrons
down to few GeV of transverse momenta.  The ECAL has to maintain high
energy resolution up to the TeV energy range in order to allow the
discovery of possible new physics resonances involving electrons and
photons, and possible candidates for the dark matter.  The ECAL is
designed with fine transverse granularity in order to maintain high
identification performances for electrons and photons with respect to
hadronic jets.

\section{Overview of the ECAL in LHC Run I and Run II}
The ECAL is located inside the CMS superconducting solenoidal magnet
and made of 61,200 PbWO$_4$ scintillating crystals, mounted in the
central barrel part (EB), closed by 7324 crystals in each of the two
endcaps (EE). The scintillation light is detected by avalanche
photodiodes in the barrel section and by vacuum phototriodes in the
two endcap sections. A preshower detector (ES), based on lead
absorbers equipped with silicon strip sensors, is installed in front
of the endcaps crystals in order to improve $\gamma$ and $\pi^0$
resolution. The ECAL geometry for the LHC Run II is expected to be the
same as in Run I, and the operation and calibrations workflow are
expected to be similar, even though new challenges arise for the
higher instantaneous luminosity and radiation doses expected for the
Run II starting in 2015.

\section{ECAL Monitoring and Calibration}


The main effect on the ECAL crystals of radiation at the LHC is a
wavelength-dependent loss of crystal transparency without change to
the scintillation mechanism.  A secondary effect of the radiation is
that the VPT response depends on the accumulated photocathode charge.
These effects are larger in the high $\eta$ regions (in particular the
endcaps) and they will be more challenging in Run II, when the
instantaneous luminosity is expected to increase up to $\approx$
1.5$\times$10$^{34}$, twice the value of Run I. The response change is
measured with a dedicated laser monitoring system. The laser signals
drop exponentially during periods of LHC operation, reaching a
saturation level which depends on the dose-rate (note that the
dose-rate increased in the different periods due to the increasing LHC
luminosity). The observed losses reached during Run I will be
presented in the full $\eta$ region covered by the ECAL. The
monitoring flow in Run II will be similar to the Run I one, with some
changes that will be presented.  The calibration procedure will be
presented, relying upon different methods: 
\begin{enumerate}
\item $\phi$-symmetry, to provide fast inter-calibration by
  exploiting the invariance around the beam axis of energy flow in
  minimum bias events,
\item $\pi^0$ and $\eta$ calibration, using the invariant mass peak
  from photon pairs,
\item electrons from $Z\to e^+e^-$ pairs and $W\to e\nu$ decays,
  comparing the energy measured in ECAL to the track momentum measured
  in the silicon tracker.
\end{enumerate}
The performances of these calibration in Run I will be presented, and
the techniques adopted to maintain them with the higher LHC rates and
pileup will be presented.

\section{ECAL energy, H$\to\gamma\gamma$ and H$\to4\ell$ resolution}
The search for the Higgs Boson in the two-photon decay channel relies
on the energy resolution and the position measurement within ECAL of
the reconstructed photons. The simulation of response to photons is
tuned in the different ECAL regions to match the one observed in situ
from $Z\to e^+e^-$ events, where electrons are reconstructed as
photons.  The relative invariant mass resolution of simulated
H$\to\gamma\gamma$ events will be presented.  The H$\to4\ell$ search,
with decays involving two or four electrons, cover an electron
momentum range from 7 GeV to $\approx$ 50 GeV, for most of which the
energy measurement is dominated by ECAL.  The simulated mass
resolution with the ECAL, or with ECAL plus tracker combination, after
the simulation tuning to match the data will be presented. A
comparison of the Higgs boson mass measured in the two decay modes
most sensitive to its signal will be presented.

\section{Conclusions}
The CMS ECAL has played a major role in the discovery of the Higgs
boson during LHC Run I. The response variations during three years of
data taking in different conditions have been monitored with the
expected precision, and allowed to maintain its outstanding design
performances. The Run II poses new challenges, in terms of the higher
rates expected for the calibration physics triggers, the higher
radiation doses that will affect the crystal transparency variations,
and the higher pileup from both the same and different bunch
crossings.

\end{multicols}


\end{document}
